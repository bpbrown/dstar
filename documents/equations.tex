\documentclass{aastex62}

\usepackage{amsthm, amsmath, amssymb}
\usepackage{latexsym,graphicx,rotating,amsmath, epsfig, natbib, graphbox}
\usepackage{listings}

\newcommand{\sol}{\odot}
\newcommand{\del}{\nabla}
\newcommand{\cross}{\times}
\newcommand{\avg}{\bar}
\renewcommand{\vec}{\boldsymbol}
\newcommand{\pomega}{\varpi}
\newcommand{\conv}{\boldsymbol}

\newcommand{\scrB}{\mathcal{B}}
\newcommand{\scrC}{\mathcal{C}}
\newcommand{\scrD}{\mathcal{D}}
\newcommand{\scrH}{\mathcal{H}}
\newcommand{\scrR}{\mathcal{R}}
\newcommand{\scrP}{\mathcal{P}}
\newcommand{\scrL}{\mathcal{L}}
\newcommand{\scrS}{\mathcal{S}}

\newcommand{\Ra}{\mathrm{Ra}}
\newcommand{\Ek}{\mathrm{Ek}}
\renewcommand{\Pr}{\mathrm{Pr}}
\newcommand{\Pm}{\mathrm{Pm}}
\newcommand{\RoCsq}{\mathrm{Ro}_\mathrm{C}^2}
\newcommand{\RoC}{\mathrm{Ro}_\mathrm{C}}

\newcommand{\expm}{\mathrm{expm1}}

\newcommand{\dedalus}{\href{http://dedalus-project.org/}{Dedalus}}

\definecolor{codegreen}{rgb}{0,0.6,0}
\definecolor{codegray}{rgb}{0.5,0.5,0.5}
\definecolor{codepurple}{rgb}{0.58,0,0.82}
\definecolor{backcolour}{rgb}{0.95,0.95,0.92}
\definecolor{codered}{rgb}{0.6,0,0}
\definecolor{codeblue}{rgb}{0,0,0.6}

% \watermark{text}
\begin{document}
\section{Basic equations}
Here we derive and non-dimensionalize a form of the fully compressible equations.  We use the enthalpy $h$ and entropy $s$ as thermodynamic variables, along with the density $\rho$ and temperature $T$ (though we swiftly convert $T$ to $h$ in this document).  We use log formulations of both $h$ and $\rho$ for better handling of high as well as low Mach situations.

Here we also take the dynamic viscous diffusion coefficient $\mu$ and thermal diffusion coefficient $K$ as appropriate models for the underlying transport processes.

The fully compressible equations, in a rotating reference frame, are:
\begin{equation}
  \partial_t \ln \rho + \vec{u} \cdot \del \ln \rho + \del \cdot \vec{u} = 0
\end{equation}
\begin{equation}
  \partial_t \vec{u} + \vec{u}\cdot \del \vec{u} + \del (h + \phi) = T\del s + \frac{1}{\rho}\del\cdot (\mu E_{ij}) - 2 \vec{\Omega} \times \vec{u}
  \label{eq:u}
\end{equation}
\begin{equation}
  \partial_t s + \vec{u}\cdot \del s = \frac{1}{\rho T}\left[\del \cdot (K \del T) + \frac{\mu}{2}\mathrm{Tr}(E^2)\right] + \frac{\epsilon_\mathrm{source}}{T}
  \label{eq:entropy}
\end{equation}
with
\begin{equation}
  T = \frac{h}{c_P}, \quad E_{ij} = \del \vec{u} + (\del \vec{u})^\mathrm{T} - \frac{2}{3}(\del\cdot\vec{u})\mathrm{I},
\end{equation}
and linked by an ideal gas equation of state:
\begin{equation}
  \frac{\gamma}{\gamma-1} \frac{s}{c_P} - \frac{1}{\gamma - 1}\ln h + \ln \rho =0
\end{equation}
The momentum equation~\ref{eq:u} comes from standard form and incorporating the thermodynamic identity:
\begin{equation}
  \mathrm{d} h = T \mathrm{d} s - v \mathrm{d} P
\end{equation}
with $\mathrm{d} \rightarrow \del$ and $v = 1/\rho$.
The entropy equation~\ref{eq:entropy} comes from \citep{Mihalas&Mihalas} eq~(27.17), with the addition of an entropy source $\epsilon/T$, like from nuclear heating or changes in radiative transport.

The gas constant $R$ has been incorporated within the definition of $T$, and the equations are fully dimensional.  In this system:
\begin{equation}
  c_P = \frac{\gamma}{\gamma - 1}
\end{equation}



\subsection{Enthalpy all the time}
The first step is to rewrite the entropy and momentum equations to be entirely in terms of enthalphy.

\subsubsection{entropy}
In the entropy equation, there are some manipulations which will make this system more clear.  The first step is to convert from $T$ to $h$, using $h = c_P T$.
\begin{equation}
  \partial_t s + \vec{u}\cdot \del s = \frac{1}{\rho h}\left[\del \cdot (K \del h)\right] + \frac{c_P \mu}{2 \rho h}\mathrm{Tr}(E^2) + \frac{c_P \epsilon_\mathrm{source}}{h}
\end{equation}

The diffusive term suggests a log enthalphy form:
\begin{align}
  \frac{1}{\rho h}\del\cdot K \del h &= \frac{K}{\rho}\left[\frac{1}{h}\del\cdot\del h + \frac{\del K \cdot \del h}{K h} \right] \\
  & = \frac{K}{\rho}\left[\nabla^2 \ln h + \left(\nabla \ln h\right)^2 + \del \ln K \cdot \del \ln h \right]
\end{align}
which results in:
\begin{equation}
  \partial_t s + \vec{u}\cdot \del s = \frac{K}{\rho}\left[\nabla^2 \ln h + \left(\nabla \ln h\right)^2 + \del \ln K \cdot \del \ln h \right] + \frac{c_P \mu}{2 \rho h}\mathrm{Tr}(E^2) + \frac{c_P \epsilon_\mathrm{source}}{h}
\end{equation}
Note that if we fully embrace $\ln h$ and $\ln \rho$ as the primary thermodynamic variables (which we will), with $h = \exp(\ln h)$, $\rho = \exp(\ln \rho)$, we obtain:
\begin{multline}
  \partial_t s + \vec{u}\cdot \del s = K(r) \exp(-\ln \rho)\left[\nabla^2 \ln h + \left(\nabla \ln h\right)^2 + \del \ln K \cdot \del \ln h \right] \\ + \frac{c_P \mu}{2} \exp(-\ln \rho) \exp(-\ln h)\mathrm{Tr}(E^2) + c_P \epsilon_\mathrm{source} \exp(-\ln h)
\end{multline}
where we are also now explicitly noting that $K=K(r)$ (which may be a constant function of $r$).

We might be alarmed at trading a $1/h$ for a $\exp{(-\ln h)}$, but we shouldn't be: $1/h$ is already a band-unlimited nonlinear term, and $\mathrm{Tr}(E^2)$ and $\epsilon$ are always going to be on the RHS.

We will have to take more care however with the diffusive term, which in the current representation is now a nonlinear term.  There are multiple reasonable approaches to this, including either taking $K(r) \exp(-\ln \rho) \approx K(r) \exp(-\ln \rho_0) = \chi(r)$, or treating this as a fully nonlinear term and using implicit/explicit stepping techniques combined with clever destiffening of the equations.

\subsubsection{momentum}
In the momentum equation, this transformation is more straightforward, with only the $T\del s$ term changing:
\begin{equation}
  \partial_t \vec{u} + \vec{u}\cdot \del \vec{u} + \del (h + \phi) = \frac{h}{c_P}\del s + \frac{1}{\rho}\del\cdot (\mu E_{ij}) - 2 \vec{\Omega} \times \vec{u}
\end{equation}
Adopting $\ln h$ and $\ln \rho$ as the primary thermodynamic variables:
\begin{equation}
  \partial_t \vec{u} + \vec{u}\cdot \del \vec{u} + \del (\exp(\ln h) + \phi) = \exp(\ln h)\frac{1}{c_P}\del s + \exp(-\ln\rho)\del\cdot (\mu E_{ij}) - 2 \vec{\Omega} \times \vec{u}
\end{equation}
Clearly we're going to have to be a bit careful with our now nonlinear enthalpy gradient term $\del \exp(\ln h)$ and our buoyancy term, which has changed from a quadratic nonlinearity to a band-unlimited nonlinearity, as well as our diffusive term with it's potential nonlinear leading coefficient.

To make progress, we will consider equilibria and then non-dimensionalize these equations.

\subsection{Hydrostatic and thermal equilibrium}
The hydrostatic momentum equations is:
\begin{equation}
  \del (\exp(\ln h) + \phi) = \exp(\ln h)\frac{1}{c_P}\del s
\end{equation}
and the thermal equilibrium equation is:
\begin{equation}
  K(r) \exp(-\ln \rho)\left[\nabla^2 \ln h + \left(\nabla \ln h\right)^2 + \del \ln K \cdot \del \ln h \right] = - c_P \epsilon_\mathrm{source} \exp(-\ln h)
\end{equation}
We now decompose the enthalpy, entropy and density into equilibrium and (potentially large) fluctuations:
\begin{align}
  \ln h = \ln h_0 + \ln h_1, &\quad \text{and} \quad \exp(\ln h) = h_0 \exp(\ln h_1),\\
  \ln \rho = \ln \rho_0 + \ln \rho_1, &\quad \text{and} \quad \exp(\ln h) = \rho_0 \exp(\ln \rho_1),\\
  s = s_0 + s_1.
\end{align}
This leads to these equilibrium equations, assuming all fluctations are zero:
\begin{equation}
  \del \ln h_0 = -\frac{\del \phi}{h_0} + \frac{1}{c_P}\del s_0
\end{equation}
and the thermal equilibrium equation is:
\begin{equation}
  K(r) \left[\nabla^2 \ln h_0 + \left(\nabla \ln h_0\right)^2 + \del \ln K \cdot \del \ln h \right] = - \frac{\rho_0}{h_0} c_P \epsilon_\mathrm{source}
\end{equation}
or
\begin{equation}
  \left[\nabla^2 h_0 + \del \ln K \cdot \del h_0 \right] = - \frac{\rho_0}{K(r) } c_P \epsilon_\mathrm{source}
  \label{eq:thermal equilibrium}
\end{equation}

The big question moving forward is whether to take a thermally equilibrated reference state, with equation~(\ref{eq:thermal equilibrium}) being satisfied, or whether to take an adiabatic reference state with $\del s_0 = 0$.  In either case, we'll take hydrostatic equilibrium.

\subsection{Fluctuation equations, with $\del s_0 = 0$}
We'll start by taking an adiabatic state ($\del s_0 = 0$) in hydrostatic equilibrium:
\begin{equation}
  \del h_0 = - \del \phi
\end{equation}

Now define:
\begin{align}
  \theta \equiv \ln h_1 \\
  \Upsilon \equiv \ln \rho_1
\end{align}

The momentum equation is:
\begin{equation}
  \partial_t \vec{u} + \vec{u}\cdot \del \vec{u} + \del (h_0[\exp(\theta)-1]) = h_0\exp(\theta)\frac{1}{c_P}\del s + \frac{1}{\rho_0}\exp(-\Upsilon)\del\cdot (\mu E_{ij}) - 2 \vec{\Omega} \times \vec{u},
\end{equation}

The entropy equation is a bit more complicated, since we have quadratic terms in the diffusive part:
\begin{align}
\nabla^2 (\ln h) + \left(\nabla \ln h\right)^2 + \del \ln K \cdot \del (\ln h)
& =
\nabla^2 (\ln h_0+\theta) + \left(\nabla (\ln h_0 + \theta) \right)^2 + \del \ln K \cdot \del (\ln h_0 + \theta) \\
& = \nabla^2 \ln h_0 + (\nabla \ln h_0)^2
+ \del \ln K \cdot \del \ln h_0  \\
&\phantom{=}+ \nabla^2 \theta + 2\left(\nabla \ln h_0 \cdot \nabla \theta \right) + \left(\nabla \theta \right)^2 + \del \ln K \cdot \del \theta
\end{align}
%\frac{K(r)}{\rho_0}\exp(-\Upsilon)\left[

\begin{equation}
  \partial_t s + \vec{u}\cdot \del s = \frac{K(r)}{\rho_0} \exp(-\Upsilon)\scrD^2(\theta) + \frac{c_P \mu}{2 \rho_0 h_0} \exp(-\Upsilon) \exp(-\theta)\mathrm{Tr}(E^2) + \scrS
\end{equation}
with:
\begin{equation}
\scrD^2(\theta) = \nabla^2 \theta + 2\left(\nabla \ln h_0 \cdot \nabla \theta \right) + \left(\nabla \theta \right)^2 + \del \ln K \cdot \del \theta
\end{equation}
and with:
\begin{equation}
\scrS = c_P \epsilon \frac{1}{h_0}\exp(-\theta) + \frac{K(r)}{h_0 \rho_0} \exp(-\Upsilon) \left[\nabla^2 h_0 + \del \ln K \cdot \del h_0 \right]
\end{equation}
or
\begin{equation}
\scrS = c_P \epsilon \frac{1}{h_0}\exp(-\theta) + \frac{K(r)}{\rho_0} \exp(-\Upsilon) \nabla^2 \ln h_0 + (\nabla \ln h_0)^2
+ \del \ln K \cdot \del \ln h_0
\end{equation}
depending on preferences about forms of NCCs (this is all RHS anyways, so it's a bit academic).

The equation of state is:
\begin{equation}
  \gamma\frac{s}{c_P} = \theta - (\gamma - 1)\Upsilon
\end{equation}

\newpage
\subsection{Non-dimensionalization, $\del s_0 = 0$}
Let's take a crack at non-dimensionalizing this mess.

Let there be a characteristic timescale $t_c = \tau$ and lengthscale $l_c = L$, with $u_c = L/\tau$, and characteristic entropy $s_c$, enthalpy $h_c$ and density $\rho_c$.  For now, for sanity, take $K=\text{const}$ and $\mu=\text{const}$.

The momentum equation is:
\begin{equation}
  \partial_t \vec{u} + \vec{u}\cdot \del \vec{u} + \frac{h_c \tau^2}{L^2}\del (h_0[\exp(\theta)-1]) = \frac{h_c \tau^2}{L^2} \left(\frac{s_c}{c_P}\right) h_0\exp(\theta)\del s + \frac{\mu \tau }{L^2 \rho_c}\frac{1}{\rho_0}\exp(-\Upsilon)\del\cdot (E_{ij}) - (2 \Omega \tau) \vec{\hat{z}} \times \vec{u},
\end{equation}

The entropy equation is:
\begin{equation}
  \partial_t s + \vec{u}\cdot \del s =
  \frac{K \tau}{c_P \rho_c L^2}\left(\frac{s_c}{c_P}\right) \frac{1}{\rho_0} \exp(-\Upsilon)\scrD^2(\theta)
  + \frac{\mu}{2 \tau \rho_c h_c}\left(\frac{s_c}{c_P}\right) \frac{1}{\rho_0 h_0}\exp(-\Upsilon) \exp(-\theta)\mathrm{Tr}(E^2)
  + \tau \left(\frac{s_c}{c_P}\right)\frac{\scrS}{c_P}
\end{equation}
with
\begin{equation}
\frac{\scrS}{c_P} = \frac{K/(c_P \rho_c)}{L^2} \left[\frac{\epsilon}{h_c} \frac{L^2}{K/(c_P \rho_c)}\frac{1}{h_0}\exp(-\theta) +  \frac{1}{\rho_0} \exp(-\Upsilon) \nabla^2 \ln h_0 + (\nabla \ln h_0)^2 \right]
\end{equation}
and
\begin{equation}
\scrD^2(\theta) = \nabla^2 \theta + 2\left(\nabla \ln h_0 \cdot \nabla \theta \right) + \left(\nabla \theta \right)^2
\end{equation}

The equation of state is:
\begin{equation}
  \left(\frac{s_c}{c_P}\right)\gamma s = \theta - (\gamma - 1)\Upsilon
\end{equation}

\subsection{Choices}
Before we do anything else, let's introduce some definitions from \citet{Mihalas&Mihalas} (their equations 28.3--28.5):
\begin{align}
  \mathrm{Pr} &= \frac{\mu c_P}{K} \\
  \mathrm{Re} &= \frac{u L}{(\mu/\rho)}\\
  \mathrm{Pe} &= \mathrm{Pr}\mathrm{Re} = \frac{u L }{(K/c_p \rho) }
\end{align}
Now it's time to make some choices.  What should we use for $s_c$ and what should we use for $\tau$?
\subsubsection{Choose: $s_c = c_P$, $\tau = L/u_c$}
\begin{equation}
  \partial_t \vec{u} + \vec{u}\cdot \del \vec{u} + \frac{h_c}{u_c^2}\del (h_0[\exp(\theta)-1]) =
  \frac{h_c}{u_c^2} h_0\exp(\theta)\del s
  + \frac{\mu }{u_c L \rho_c}\frac{1}{\rho_0}\exp(-\Upsilon)\del\cdot (E_{ij})
  - \left(\frac{2 \Omega L}{u_c}\right) \vec{\hat{z}} \times \vec{u},
\end{equation}

\begin{equation}
  \partial_t s + \vec{u}\cdot \del s =
  \frac{K}{c_P \rho_c u_c L} \frac{1}{\rho_0} \exp(-\Upsilon)\scrD^2(\theta)
  + \frac{\mu}{2 \rho_c u_c L}\frac{u_c^2}{h_c}\frac{1}{\rho_0 h_0}\exp(-\Upsilon) \exp(-\theta)\mathrm{Tr}(E^2)
  + \frac{L}{u_c}\frac{\scrS}{c_P}
\end{equation}

\begin{equation}
  \gamma s = \theta - (\gamma - 1)\Upsilon
\end{equation}

Now, let $\scrR = u_c L \rho_c/\mu$ and $\scrP = u_c L c_P \rho_c /K = \mathrm{Pr}\scrR$.  Further, let $(\gamma-1)\mathrm{Ma} = (\gamma-1) u_c^2/\gamma T_c = u_c^2/h_c$ and let $\mathrm{Ro} = 2 \Omega L/u_c$.

\begin{equation}
  \partial_t \vec{u} + \vec{u}\cdot \del \vec{u} + \frac{1}{\gamma-1}\mathrm{Ma}^{-2} \del (h_0[\exp(\theta)-1]) =
  \frac{1}{\gamma-1}\mathrm{Ma}^{-2} h_0\exp(\theta)\del s
  + \frac{1}{\scrR}\frac{1}{\rho_0}\exp(-\Upsilon)\del\cdot (E_{ij})
  - \mathrm{Ro} \vec{\hat{z}} \times \vec{u},
\end{equation}

\begin{equation}
  \partial_t s + \vec{u}\cdot \del s =
  \frac{1}{\scrP} \frac{1}{\rho_0} \exp(-\Upsilon)\scrD^2(\theta)
  + \frac{1}{\scrR}(\gamma-1)\mathrm{Ma}^2 \frac{1}{\rho_0 h_0}\frac{1}{2}\exp(-\Upsilon) \exp(-\theta)\mathrm{Tr}(E^2)
  + \frac{L}{u_c}\frac{\scrS}{c_P}
\end{equation}


\subsubsection{Choose: $s_c = c_P$, $\tau = 1/(2\Omega)$}

\begin{equation}
  \partial_t \vec{u} + \vec{u}\cdot \del \vec{u} + \frac{h_c}{(2\Omega L)^2}\del (h_0[\exp(\theta)-1]) = \frac{h_c}{(2\Omega L)^2} h_0\exp(\theta)\del s + \frac{\mu/\rho_c}{2 \Omega L^2}\frac{1}{\rho_0}\exp(-\Upsilon)\del\cdot (E_{ij}) - \vec{\hat{z}} \times \vec{u},
\end{equation}

The entropy equation is:
\begin{equation}
  \partial_t s + \vec{u}\cdot \del s =
  \frac{K/(c_P \rho_c)}{2 \Omega L^2} \frac{1}{\rho_0} \exp(-\Upsilon)\scrD^2(\theta)
  + \frac{\mu/\rho_c}{2\Omega L^2} \frac{(2\Omega L)^2}{h_c} \frac{1}{2 }\frac{1}{\rho_0 h_0}\exp(-\Upsilon) \exp(-\theta)\mathrm{Tr}(E^2)
  + \frac{1}{2 \Omega} \frac{\scrS}{c_P}
\end{equation}

And if we take $\mathrm{Ek} = (\mu/\rho_c)/(2 \Omega L^2)$ then these look like:

\begin{equation}
  \partial_t \vec{u} + \vec{u}\cdot \del \vec{u} + \left(\frac{h_c}{(2\Omega L)^2}\right)\del (h_0[\exp(\theta)-1]) = \left(\frac{h_c}{(2\Omega L)^2}\right) h_0\exp(\theta)\del s + \mathrm{Ek}\frac{1}{\rho_0}\exp(-\Upsilon)\del\cdot (E_{ij}) - \vec{\hat{z}} \times \vec{u},
\end{equation}

The entropy equation is:
\begin{equation}
  \partial_t s + \vec{u}\cdot \del s =
  \frac{\mathrm{Ek}}{\mathrm{Pr}} \frac{1}{\rho_0} \exp(-\Upsilon)\scrD^2(\theta)
  + \mathrm{Ek} \left(\frac{(2\Omega L)^2}{h_c}\right) \frac{1}{2}\frac{1}{\rho_0 h_0}\exp(-\Upsilon) \exp(-\theta)\mathrm{Tr}(E^2)
  + \frac{1}{2 \Omega} \frac{\scrS}{c_P}
\end{equation}

We're left with an interesting term in these equations: $\left(h_c/(2\Omega L)^2\right)$.  Let's see what we can turn that into.
\begin{equation}
\frac{h_c}{(2\Omega L)^2}
= \frac{h_c L^2}{(\mu/\rho_c)^2} \frac{(\mu/\rho_c)^2}{(2\Omega L^2)^2}
= \frac{h_c L^2}{(\mu/\rho_c)(K/(c_P \rho_c))}\left(\frac{K/(c_P \rho_c)}{\mu/\rho_c}\right) \left(\frac{(\mu/\rho_c)}{2\Omega L^2}\right)^2
= \frac{\mathrm{R}^* \mathrm{Ek}^2}{\mathrm{Pr}}
\end{equation}
which suggests the new definition:
\begin{equation}
\mathrm{R}^* = \frac{h_c L^2}{(\mu/\rho_c)(K/(c_P \rho_c))}
\end{equation}
Oh, interesting!  Now adopt our earlier $\scrR = u_c L/(\mu/\rho_c)$ and $\scrP = u_c L/(K/c_P \rho_c) = \mathrm{Pr}\scrR$:
\begin{equation}
\mathrm{R}^* = \frac{h_c}{u_c^2}\frac{u_c^2 L^2}{(\mu/\rho_c)(K/(c_P \rho_c))} = \frac{1}{\gamma-1}\frac{1}{\mathrm{Ma}^{2}} \scrR \scrP
\end{equation}
and if we take
\begin{equation}
  \mathrm{Ra} = \scrR \scrP = \frac{u_c^2 L^2}{(\mu/\rho_c)(K/(c_P \rho_c))}
\end{equation}
then:
\begin{equation}
  \frac{h_c}{(2\Omega L)^2} = \frac{1}{\gamma-1}\frac{1}{\mathrm{Ma}^{2}}\frac{\mathrm{Ra}\mathrm{Ek}^2}{\mathrm{Pr}}
\end{equation}
where throughout:
\begin{equation}
  \mathrm{Ma}^{2} \equiv \frac{u_c^2}{\gamma T_c}
\end{equation}
namely the adiabatic Mach number.

Clean this up by defining:
\begin{equation}
  \mathrm{Co}^2 = \frac{\mathrm{Ra}\mathrm{Ek}^2}{\mathrm{Pr}}
\end{equation}
and:
\begin{equation}
  \frac{h_c}{(2\Omega L)^2} = \frac{1}{\gamma-1}\frac{\mathrm{Co}^2}{\mathrm{Ma}^{2}}
\end{equation}

and we end with:

\begin{equation}
  \partial_t \vec{u} + \vec{u}\cdot \del \vec{u} + \frac{1}{\gamma-1}\frac{\mathrm{Co}^2}{\mathrm{Ma}^{2}}\del (h_0[\exp(\theta)-1]) =
  \frac{1}{\gamma-1}\frac{\mathrm{Co}^2}{\mathrm{Ma}^{2}} h_0\exp(\theta)\del s
  + \mathrm{Ek}\frac{1}{\rho_0}\exp(-\Upsilon)\del\cdot (E_{ij}) - \vec{\hat{z}} \times \vec{u},
\end{equation}

The entropy equation is:
\begin{equation}
  \partial_t s + \vec{u}\cdot \del s =
  \frac{\mathrm{Ek}}{\mathrm{Pr}} \frac{1}{\rho_0} \exp(-\Upsilon)\scrD^2(\theta)
  + \mathrm{Ek} (\gamma-1)\frac{\mathrm{Ma}^{2}}{\mathrm{Co}^2} \frac{1}{2}\frac{1}{\rho_0 h_0}\exp(-\Upsilon) \exp(-\theta)\mathrm{Tr}(E^2)
  + \frac{1}{2 \Omega} \frac{\scrS}{c_P}
\end{equation}

\subsubsection{Grappling with the source term}
Time to finally grapple with the source term.

\begin{align}
  \frac{1}{2 \Omega} \frac{\scrS}{c_P}
&= \frac{K/(c_P \rho_c)}{2 \Omega L^2} \left[\frac{\epsilon}{h_c} \frac{L^2}{K/(c_P \rho_c)}\frac{1}{h_0}\exp(-\theta) +  \frac{1}{\rho_0} \exp(-\Upsilon) \left(\nabla^2 \ln h_0 + (\nabla \ln h_0)^2 \right)\right] \nonumber \\
&= \frac{\mathrm{Ek}}{\mathrm{Pr}}\left[\frac{\epsilon}{h_c} \frac{L^2}{K/(c_P \rho_c)}\frac{1}{h_0}\exp(-\theta) +  \frac{1}{\rho_0} \exp(-\Upsilon) \left(\nabla^2 \ln h_0 + (\nabla \ln h_0)^2 \right)\right]
\end{align}

What do we do now?  Let's consider thermal and hydrostatic equilibrium for a moment.  In that case:
\begin{align}
\frac{\epsilon}{h_c} \frac{L^2}{K/(c_P \rho_c)}\frac{1}{h_0} +  \frac{1}{\rho_0} \left(\nabla^2 \ln h_0 + (\nabla \ln h_0)^2 \right) = 0
\end{align}
From the equation of state:
\begin{equation}
  \gamma \nabla^2 s_0 = \nabla^2 \ln h_0 - (\gamma-1)\nabla \ln \rho_0
\end{equation}
or:
\begin{align}
\frac{\epsilon}{h_c} \frac{L^2}{K/(c_P \rho_c)}\frac{1}{h_0} +  \frac{1}{\rho_0} \left(\gamma \nabla^2 s_0 + (\gamma-1)\nabla \ln \rho_0 + (\nabla \ln h_0)^2 \right) = 0
\end{align}
and:
\begin{align}
 \gamma \nabla^2 s_0 = - \rho_0 \frac{\epsilon}{h_c} \frac{L^2}{K/(c_P \rho_c)}\frac{1}{h_0} - (\gamma-1)\nabla \ln \rho_0 - (\nabla \ln h_0)^2
\end{align}
and taking:
\begin{equation}
  \Delta s_0 \sim L^2 \nabla^2 s_0
\end{equation}
we get:
\begin{align}
 \Delta s_0 \sim - \frac{L^2}{\gamma} \frac{\epsilon}{h_c} \frac{L^2}{K/(c_P \rho_c)}\frac{\rho_0}{h_0}
\end{align}
It would clearly be nifty if $\Delta s_0$ was approximately constant as we change for example $K$, and this suggests $\epsilon$ scale with $K$ and that we take
\begin{equation}
Q q(r)\equiv \frac{L^2}{\gamma} \frac{\epsilon(r)}{h_c} \frac{L^2}{K/(c_P \rho_c)}
\end{equation}
with $q(r)$ the shape function of the heating,
in which case we get
\begin{align}
 \Delta s_0 \sim - Q\frac{q(r) \rho_0}{h_0}
\end{align}
and we expect that $\mathrm{Ma}^2 \propto \Delta s_0 \propto Q$ and we now have a control parameter for adjusting the Mach number.  Maybe.

Well, let's go with it for now.

\subsection{Equations from on high}

\begin{equation}
  \partial_t \vec{u} + \vec{u}\cdot \del \vec{u} + \frac{1}{\gamma-1}\frac{\mathrm{Co}^2}{\mathrm{Ma}^{2}}\del (h_0[\exp(\theta)-1]) =
  \frac{1}{\gamma-1}\frac{\mathrm{Co}^2}{\mathrm{Ma}^{2}} h_0\exp(\theta)\del s
  + \mathrm{Ek}\frac{1}{\rho_0}\exp(-\Upsilon)\del\cdot (E_{ij}) - \vec{\hat{z}} \times \vec{u},
\end{equation}

\begin{align}
  \partial_t s + \vec{u}\cdot \del s &=
  \frac{\mathrm{Ek}}{\mathrm{Pr}} \frac{1}{\rho_0} \exp(-\Upsilon)\scrD^2(\theta) \nonumber\\
  &\phantom{=}+ \mathrm{Ek} (\gamma-1)\frac{\mathrm{Ma}^{2}}{\mathrm{Co}^2} \frac{1}{2}\frac{1}{\rho_0 h_0}\exp(-\Upsilon) \exp(-\theta)\mathrm{Tr}(E^2)\nonumber\\
  &\phantom{=}+ \frac{\mathrm{Ek}}{\mathrm{Pr}}\left[Q \frac{q(r)}{h_0}\exp(-\theta) +  \frac{1}{\rho_0} \exp(-\Upsilon) \left(\nabla^2 \ln h_0 + (\nabla \ln h_0)^2 \right)\right]
\end{align}

\subsection{De-stiffening acoustic modes}
Now, let's use our experience with de-stiffening these equations, and let's put these in implicit/explicit form.  First, let's clean things up by letting:
\begin{equation}
  \scrC \equiv \frac{1}{\gamma-1}\frac{\mathrm{Co}^2}{\mathrm{Ma}^{2}}
\end{equation}

\begin{multline}
  \partial_t \vec{u}
  + \scrC \del (h_0 \theta)
  - \scrC h_0 \del s
  - \mathrm{Ek}\frac{1}{\rho_0}\del\cdot (E_{ij})
  =
  - \vec{u}\cdot \del \vec{u} - \vec{\hat{z}} \times \vec{u} \\
  - \scrC \del (h_0[\exp(\theta)-\theta - 1])
  + \scrC h_0\left[\exp(\theta)-1\right]\del s
  + \mathrm{Ek}\frac{1}{\rho_0}\left[\exp(-\Upsilon)-1\right]\del\cdot (E_{ij}),
\end{multline}
or, adopting the $\expm = \exp - 1$ operator:
\begin{multline}
  \partial_t \vec{u}
  + \scrC \del (h_0 \theta)
  - \scrC h_0 \del s
  - \mathrm{Ek}\frac{1}{\rho_0}\del\cdot (E_{ij})
  =
  - \vec{u}\cdot \del \vec{u} - \vec{\hat{z}} \times \vec{u} \\
  - \scrC \del (h_0[\expm(\theta)-\theta])
  + \scrC h_0 \expm(\theta)\del s
  + \mathrm{Ek}\frac{1}{\rho_0}\expm(-\Upsilon) \del\cdot (E_{ij}),
\end{multline}

\begin{align}
  \partial_t s
  - \frac{\mathrm{Ek}}{\mathrm{Pr}} \frac{1}{\rho_0} \left[\nabla^2 \theta + 2\left(\nabla \ln h_0 \cdot \nabla \theta \right)\right]
   &= - \vec{u}\cdot \del s \nonumber \\
  &\phantom{=}+ \frac{\mathrm{Ek}}{\mathrm{Pr}} \frac{1}{\rho_0} \expm(-\Upsilon)\left[\nabla^2 \theta + 2\left(\nabla \ln h_0 \cdot \nabla \theta \right)\right]
  + \frac{\mathrm{Ek}}{\mathrm{Pr}} \frac{1}{\rho_0} \exp(-\Upsilon) \left[\left(\nabla \theta \right)^2\right] \nonumber\\
  &\phantom{=}+  \frac{\mathrm{Ek}}{\scrC} \frac{1}{2}\frac{1}{\rho_0 h_0}\exp(-\Upsilon) \exp(-\theta)\mathrm{Tr}(E^2)\nonumber\\
  &\phantom{=}+ \frac{\mathrm{Ek}}{\mathrm{Pr}}\left[Q \frac{q(r)}{h_0}\exp(-\theta) +  \frac{1}{\rho_0} \exp(-\Upsilon) \left(\nabla^2 \ln h_0 + (\nabla \ln h_0)^2 \right)\right]
\end{align}

\subsection{Now, ignore nonlinearities in diffusion and 1/h terms}
\begin{multline}
  \partial_t \vec{u}
  + \scrC \del (h_0 \theta)
  - \scrC h_0 \del s
  - \mathrm{Ek}\frac{1}{\rho_0}\del\cdot (E_{ij})
  =
  - \vec{u}\cdot \del \vec{u} - \vec{\hat{z}} \times \vec{u}
  - \scrC \del (h_0[\expm(\theta)-\theta])
  + \scrC h_0 \expm(\theta)\del s
\end{multline}

\begin{align}
  \partial_t s
  - \frac{\mathrm{Ek}}{\mathrm{Pr}} \frac{1}{\rho_0} \left[\nabla^2 \theta + 2\left(\nabla \ln h_0 \cdot \nabla \theta \right)\right]
   &= - \vec{u}\cdot \del s +
  \frac{\mathrm{Ek}}{\mathrm{Pr}} \frac{1}{\rho_0} \left[\left(\nabla \theta \right)^2\right]
  +  \frac{\mathrm{Ek}}{\scrC} \frac{1}{2}\frac{1}{\rho_0 h_0}\mathrm{Tr}(E^2)\nonumber\\
  &\phantom{=}+ \frac{\mathrm{Ek}}{\mathrm{Pr}}\left[Q \frac{q(r)}{h_0} +  \frac{1}{\rho_0} \left(\nabla^2 \ln h_0 + (\nabla \ln h_0)^2 \right)\right]
\end{align}


\section{Being smarter}
I think we could do better by considering:
\begin{equation}
  \epsilon = \del \cdot \vec{q}
\end{equation}
and considering the flux associated with the heating.  But that's for later.

\end{document}
























\subsection{Hydrostatic equilibrium in a polytrope}
If we assert hydrostatic equilbrium, we have:
\begin{equation}
  \del (h + \phi) = h\frac{\del s}{c_P}
\end{equation}
or
\begin{equation}
  \del \ln h  - \frac{\del s}{c_P} = - \frac{\del \phi}{h}
\end{equation}
For an ideal gas we take
\begin{equation}
  \frac{\del s}{c_P} = \frac{1}{\gamma}\del \ln h - \frac{\gamma-1}{\gamma}\del \ln \rho
\end{equation}
and for a polytrope, we take
\begin{equation}
  \ln \rho = m \ln h
\end{equation}
in place of the thermal equation.

If we take $L = R T_c/\phi_c$, or the density scale height of the equivalent isothermal atmosphere at the characteristic level (say the bottom of the atmosphere), take constant gravity, and take $s_c = c_P$, then we get this non-dimensional form:
\begin{equation}
  \del \ln h  - \del s = - \frac{\gamma-1}{\gamma}\exp{(-\ln h)} \vec{\hat{z}}
\end{equation}
where the $c_P$ in the enthalpy has been factored to pull out the $R$ separately from the $(\gamma-1)/\gamma$

You know, I don't think there is actually any lengthscale information here.  What we have is $R T_c/\phi_c = 1$, because if L is the characteristic lengthscale for $\del \phi$, then the L's cancel out.  Does that somehow imply that $L=H_\rho$?  Or that $L=L_z$, the depth of the domain.  Alternatively, the correct scale is $\del \phi = \vec{g}$ and then we have a lengthscale $ L = R T_c/g_c = H_\rho$.  Very confusing.

I wonder also if there's anything in this ratio:
\begin{equation}
  \frac{\phi_c}{h_c}~\text{or}~L=\frac{g}{h_c}
\end{equation}
I guess that's an adiabatic enthalphy scale height?
(based on:)
\begin{equation}
  \del \ln h = - \frac{\del \phi}{h}, \quad\text{with} \quad \del s = 0.
\end{equation}
For now we stick to a density scaleheight system, just because.

Here's some code that solves this above problem:
\lstset{language=Python,
  backgroundcolor=\color{backcolour},
  commentstyle=\color{codered},
  keywordstyle=\color{codepurple},
  numberstyle=\tiny\color{codegray},
  stringstyle=\color{codepurple},
  frame=lines,
  breakatwhitespace=false,
  breaklines=true,
  captionpos=b,
  keepspaces=true,
  numbers=left,
  numbersep=5pt,
  showspaces=false,
  showstringspaces=false,
  showtabs=false,
  tabsize=2,
  inputencoding=utf8,
  extendedchars=true,
  literate={Υ}{{$\Upsilon$}}1 {θ}{{$\theta$}}1 {τ}{{$\tau$}}1 {φ}{{$\phi$}}1 {γ}{{$\gamma$}}1,
  }
\begin{lstlisting}
HS_problem = problems.NLBVP([Υ, θ, S, τθ])
HS_problem.add_equation((grad(θ) - grad(S) + τθ*P1 ,
                         -1*exp(-θ)*grad_φ*ez))
HS_problem.add_equation((Υ - m*θ, 0))
HS_problem.add_equation((S - 1/γ*θ+(γ-1)/γ*Υ, 0))
HS_problem.add_equation((θ(z=0),0))
\end{lstlisting}

\subsection{Exact solutions to polytropes}
The exact solution to a polytrope is:
\begin{equation}
  h(z) = h(z=0) + h_z z
\end{equation}
where
\begin{equation}
  h(z=0) =
  \begin{cases}
    1, \text{if}\ z_c=0\\
    1+L_z h_z, \text{if}\ z_c=L_z
  \end{cases}
\end{equation}
and if $L = (h/g)(z=z_c) = H_h(z=z_c)$:
\begin{equation}
\del_\phi = 1 \quad\text{and}\quad
h_z = -\frac{\gamma}{\gamma-1}\frac{1}{(m+1)}
\end{equation}
but if $L = (RT/g)(z=z_c) = H_\rho(z=z_c)$:
\begin{equation}
\del_\phi = \frac{(\gamma-1)}{\gamma} \quad\text{and}\quad
h_z = -\frac{1}{(m+1)}
\end{equation}


\section{A thermal equilibrum state}
Let's now assert a base state, assuming hydrostatic equilibrium and a thermal equilibrium.

Thermal equilibrium for this system is:
\begin{equation}
\left(\partial_t \frac{s_0}{c_P}\right) = \kappa\exp{(-\ln \rho_0)}\left[\nabla^2 \ln h_0 + \left(\nabla \ln h_0\right)^2\right] = 0
\end{equation}
or
\begin{equation}
\nabla^2 \ln h_0 + \left(\nabla \ln h_0\right)^2 = 0
\end{equation}

Recognizing that:
\begin{equation}
  \exp{(\ln h)} = \exp{(\ln h_0)}\exp{(\ln h_1)} = h_0 \exp{(\ln h_1)}
\end{equation}
and that:
\begin{equation}
  \del \left[h_0 \exp{(\ln h_1)} + \phi\right] = \del \left[h_0 (\exp{(\ln h_1)}-1) +h_0 + \phi\right]
\end{equation}
combined with:
\begin{align}
    h \frac{\del s}{c_P}
    &= h_0 \exp{(\ln h_1)}\frac{\del(s_0+s_1)}{c_P} \\
    &= h_0 \frac{\del s_0}{c_P}
    + h_0 \frac{\del s_1}{c_P}
    + h_0 \left[\exp{(\ln h_1)}-1\right]\frac{\del s_0}{c_P}
    + h_0 \left[\exp{(\ln h_1)}-1\right]\frac{\del s_1}{c_P}
\end{align}
with hydrostatic equilbrium:
\begin{equation}
  \del \left[h_0 + \phi\right] = h_0 \frac{\del s_0}{c_P}
\end{equation}
we have our pressure/buoyancy balance:
\begin{equation}
  \del\left[h_0\left(\exp{(\ln h_1)}-1\right)\right]
  = h_0 \frac{\del s_1}{c_P}
  + h_0 \left[\exp{(\ln h_1)}-1\right]\frac{\del s_0}{c_P}
  + h_0 \left[\exp{(\ln h_1)}-1\right]\frac{\del s_1}{c_P}.
\end{equation}

Let's work a bit further to pull out all linear-like terms.
First, let's take $\theta = \ln h_1$ and absorb $c_P$ into $s$:
\begin{equation}
  \del\left[h_0\left(\exp{(\theta)}-1\right)\right]
  = h_0 \del s_1
  + h_0 \left[\exp{(\theta)}-1\right] \del s_0
  + h_0 \left[\exp{(\theta)}-1\right] \del s_1.
\end{equation}
Now we add/subtract $\theta$ terms to de-stiffen $\exp{\theta}$ when linear terms can be isolated:
\begin{align}
  \del\left[h_0 \theta\right] + \del\left[h_0\left(\exp{(\theta)} -1 -\theta \right)\right]
  = h_0 \del s_1
  + h_0 \theta \del s_0
  + h_0 \left[\exp{(\theta)} -1 -\theta\right] \del s_0
  + h_0 \left[\exp{(\theta)} -1 \right] \del s_1.
\end{align}
Now we collect linear terms on the left, and nonlinear terms on the right:
\begin{equation}
  \del\left[h_0 \theta\right] - h_0 \del s_1 - (h_0 \del s_0)\theta
  =
  - \del\left[h_0\left(\exp{(\theta)} -1 -\theta \right)\right]
  + h_0 \left[\exp{(\theta)} -1 -\theta\right] \del s_0
  + h_0 \left[\exp{(\theta)} -1 \right] \del s_1.
\end{equation}

Collectively, this leads to a momentum equation with:
\begin{multline}
  \partial_t \vec{u}  +
  \del\left[h_0 \theta\right] - (h_0 \del s_0)\theta - h_0 \del s_1
  = \\
  - \vec{u}\cdot \del\vec{u}
  - \del\left[h_0\left(\exp{(\theta)} -1 -\theta \right)\right]
  + (h_0 \del s_0)\left[\exp{(\theta)} -1 -\theta\right]
  + h_0 \left[\exp{(\theta)} -1 \right] \del s_1
  + \frac{\mu}{\rho}\vec{\scrD}_1\cdot(E)
\end{multline}


\emph{A note on computation:} the $\left(\exp{(\ln h_1)}-1\right)$ term should be computed using \verb+numpy.expm1(x)+ rather than computing \verb+numpy.exp(x)+ and then subtracting 1, especially when $x \sim \mathrm{Ma}^2 \ll 1$ (numerical convergence issues).

\textbf{To do: add np.expm1 to the UnaryGridFunction list.}

With the $\expm$ operator:
\begin{multline}
  \partial_t \vec{u}  +
  \del\left[h_0 \theta\right] - (h_0 \del s_0)\theta - h_0 \del s_1
  = \\
  - \vec{u}\cdot \del\vec{u}
  - \del\left[h_0\left(\expm{(\theta)} -\theta \right)\right]
  + (h_0 \del s_0)\left[\expm{(\theta)} -\theta\right]
  + h_0 \left[\expm{(\theta)}\right] \del s_1
  + \frac{\mu}{\rho}\vec{\scrD}_1\cdot(E)
\end{multline}
We've moved various linear terms to the LHS and subtracted them off the corresponding nonlinearity on the RHS, which may help in stability especially at low Mach numbers.


\subsection{Non-dimensional form}
If we non-dimensionalize on a characteristic velocity $u_c$, then we're seeing the following non-dimensional parameter:
\begin{equation}
  \frac{h_c}{u_c^2} =
  \left(\frac{\gamma}{\gamma-1}\right)\left(\frac{R T}{u_c^2}\right) =
  \left(\frac{\gamma}{\gamma-1}\right) \mathrm{Ma}^2 =
  c_P \mathrm{Ma}^2
\end{equation}
for an isothermal Mach number $\mathrm{Ma}$.

If we also take a characteristic lengthscale $L_c = H_\rho$, and resulting timescale $\tau = L_c/u_c$, our time-derivative terms are of dimensional scale $u_c/\tau \partial_t = u_c^2/L_c \partial_t$.
This suggests:
\begin{align}
  \partial_t \vec{u}  +
  c_P \mathrm{Ma}^2 \del\left[h_0 \theta\right] &- c_P \mathrm{Ma}^2 (h_0 \del s_0)\theta - c_P \mathrm{Ma}^2 h_0 \del s_1
  = \nonumber \\
  &- \vec{u}\cdot \del\vec{u}
  - c_P \mathrm{Ma}^2 \del\left[h_0\left(\expm{(\theta)} \nonumber -\theta \right)\right] \nonumber \\
  & + c_P \mathrm{Ma}^2 (h_0 \del s_0)\left[\expm{(\theta)} -\theta\right]
  + c_P \mathrm{Ma}^2 h_0 \left[\expm{(\theta)}\right] \del s_1 \nonumber \\
  &+ \scrR \frac{1}{\rho_0}\exp{(-\ln \rho_1)}\vec{\scrD}_1\cdot(E).
\end{align}
The last term is
\begin{equation}
  \scrR = \frac{\mu}{\rho_c u_c L_c}.
\end{equation}
If we play the same linearization games with the diffusion term:
\begin{equation}
\scrR \frac{1}{\rho_0}\exp{(-\ln \rho_1)}\vec{\scrD}_1\cdot(E)
= \scrR \frac{1}{\rho_0}\vec{\scrD}_1\cdot(E)
\scrR \frac{1}{\rho_0}\left[\exp{(-\ln \rho_1)}-1\right]\vec{\scrD}_1\cdot(E).
\end{equation}
Taking $\Upsilon = \ln \rho_1$:
\begin{align}
  \partial_t \vec{u}  +
  c_P \mathrm{Ma}^2 \del\left[h_0 \theta\right] &- c_P \mathrm{Ma}^2 (h_0 \del s_0)\theta - c_P \mathrm{Ma}^2 h_0 \del s_1
  - \scrR \frac{1}{\rho_0}\vec{\scrD}_1\cdot(E) = \nonumber \\
  &- \vec{u}\cdot \del\vec{u}
  - c_P \mathrm{Ma}^2 \del\left[h_0\left(\expm{(\theta)} \nonumber -\theta \right)\right] \nonumber \\
  & + c_P \mathrm{Ma}^2 (h_0 \del s_0)\left[\expm{(\theta)} -\theta\right]
  + c_P \mathrm{Ma}^2 h_0 \left[\expm{(\theta)}\right] \del s_1 \nonumber \\
  &+ \scrR \frac{1}{\rho_0}\left[\expm{(\Upsilon)}\right]\vec{\scrD}_1\cdot(E).
\end{align}

\end{document}

\newpage

Collectively, this leads to a momentum equation with:
\begin{equation}
  \partial_t \vec{u} + \vec{u}\cdot \del\vec{u} + \del\left[h_0\left(\exp{(\ln h_1)}-1\right)\right] = \frac{h_0}{c_P}\exp{(\ln h_1)}\del s + \frac{h_0}{c_P}\left(\exp{(\ln h_1)}-1\right)\del s_0 + \frac{\mu}{\rho}\vec{\scrD}_1\cdot(E),
\end{equation}
or
\begin{equation}
  \partial_t \vec{u} + \vec{u}\cdot \del\vec{u} + \del\left[h_0\left(\exp{(\ln h_1)}-1\right)\right] -h_0 \ln h_1 \frac{\del s_0}{c_P} = h_0\exp{(\ln h_1)}\frac{\del s}{c_P} + h_0\left(\exp{(\ln h_1)}-1-\ln h_1\right)\frac{\del s_0}{c_P} + \frac{\mu}{\rho}\vec{\scrD}_1\cdot(E),
\end{equation}
or, going to $\theta = \ln h_1$ and absorbing $c_P$ into $s$:
\begin{multline}
  \partial_t \vec{u}  + \del\left[h_0\theta\right] - h_0\del s - (h_0 \del s_0) \theta  = \\
  - \vec{u}\cdot \del\vec{u} - \del\left[h_0\left(\exp{(\theta)}-1-\theta\right)\right] + h_0\left(\exp{(\theta)}-1\right)\del s
  + h_0\left(\exp{(\theta)}-1-\theta\right)\del s_0 + \frac{\mu}{\rho}\vec{\scrD}_1\cdot(E).
\end{multline}
We've moved various linear terms to the LHS and subtracted them off the corresponding nonlinearity on the RHS, which may help in stability especially at low Mach numbers.

\subsection{non-dimensional form}
If we non-dimensionalize on a characteristic velocity $u_c$, then we're seeing the following non-dimensional parameter:
\begin{equation}
  \frac{h_c}{u_c^2} =
  \left(\frac{\gamma}{\gamma-1}\right)\left(\frac{R T}{u_c^2}\right) =
  \left(\frac{\gamma}{\gamma-1}\right) \mathrm{Ma}^2 =
  c_P \mathrm{Ma}^2
\end{equation}
for an isothermal Mach number $\mathrm{Ma}$.  This suggests:
\begin{align}
\partial_t \vec{u} + &c_P \mathrm{Ma}^2 \left(\del\left[h_0\theta\right] -h_0\del s - (h_0 \del s_0) \theta \right)  =
 -  \vec{u}\cdot \del\vec{u} + \frac{\mu}{\rho}\vec{\scrD}_1\cdot(E) \\
&\phantom{=} - c_P \mathrm{Ma}^2\left(\del\left[h_0\left(\exp{(\theta)}-1-\theta\right)\right] - h_0\left(\exp{(\theta)}-1\right)\del s
- h_0\left(\exp{(\theta)}-1-\theta\right)\del s_0 \right) \nonumber
\end{align}

\emph{A note on computation:} the $\left(\exp{(\ln h_1)}-1\right)$ term should be computed using \verb+numpy.expm1(x)+ rather than computing \verb+numpy.exp(x)+ and then subtracting 1, especially when $x \sim \mathrm{Ma}^2 \ll 1$ (numerical convergence issues).

\textbf{To do: add np.expm1 to the UnaryGridFunction list.}

With the $\expm$ operator:
\begin{align}
\partial_t \vec{u} + &c_P \mathrm{Ma}^2 \left(\del\left[h_0\theta\right] -h_0\del s - (h_0 \del s_0) \theta \right)  =
-  \vec{u}\cdot \del\vec{u} + \frac{\mu}{\rho}\vec{\scrD}_1\cdot(E) \\
&\phantom{=} - c_P \mathrm{Ma}^2\left(\del\left[h_0\left(\expm{(\theta)}-\theta\right)\right] - h_0\expm{(\theta)}\del s
- h_0\left(\expm{(\theta)}-\theta\right)\del s_0 \right) \nonumber
\end{align}


\section{Here we try an adiabatic state}

\subsection{Some kind of equilibrium}
Let's now assert a base state, assuming hydrostatic equilibrium and an adiabatic profile.  This differs from assuming thermal equilibrium.
Let:
\begin{equation}
  \del s_0 = 0
\end{equation}
and
\begin{equation}
  \del(h_{0,0}\exp{\ln h_0} + \phi) = 0
\end{equation}
this means:
\begin{equation}
  h_{0,0}\exp{\ln h_0} + \phi = h_0 + \phi = \scrH
\end{equation}
for some gauge constant $\scrH$.

\subsection{Momentum equation about HS eq}

What's the momentum equation look like for fluctuations about this hydrostatic, adiabatic equilibrium?

Recognizing that:
\begin{equation}
  \exp{(\ln h)} = \exp{(\ln h_0)}\exp{(\ln h_1)} = h_0 \exp{(\ln h_1)}
\end{equation}
and that:
\begin{equation}
  \del \left[h_0 \exp{(\ln_1)} + \phi\right] = \del\left[h_0\left(\exp{(\ln h_1)}-1\right)\right]
\end{equation}
we have:
\begin{equation}
  \partial_t \vec{u} + \vec{u}\cdot \del\vec{u} + \del\left[h_0\left(\exp{(\ln h_1)}-1\right)\right] = \frac{h_0}{c_P}\exp{(\ln h_1)}\del s + \vec{\scrD}_1\cdot(\nu E),
\end{equation}
If we non-dimensionalize on a characteristic velocity $u_c$, then we're seeing the following non-dimensional parameter:
\begin{equation}
  \frac{h_0}{u_c^2} = \left(\frac{\gamma}{\gamma-1}\right)\left(\frac{R T}{u_c^2}\right) = \left(\frac{\gamma}{\gamma-1}\right) \mathrm{Ma}^2
\end{equation}
for an isothermal Mach number $\mathrm{Ma}$.  This suggests:
\begin{equation}
  \partial_t \vec{u} + \vec{u}\cdot \del\vec{u} + \left(\frac{\gamma}{\gamma-1}\right) \mathrm{Ma}^2 \del\left[h_0\left(\exp{(\ln h_1)}-1\right)\right] = \mathrm{Ma}^2 h_0 \exp{(\ln h_1)}\del s + \vec{\scrD}_1\cdot(\nu E),
\end{equation}
with $h_0$ now scaled to some reference value.

\emph{A note on computation:} the $\left(\exp{(\ln h_1)}-1\right)$ term should be computed using \verb+numpy.expm1(x)+ rather than computing \verb+numpy.exp(x)+ and then subtracting 1, especially when $x \sim \mathrm{Ma}^2 \ll 1$ (numerical convergence issues).

\textbf{To do: add np.expm1 to the UnaryGridFunction list.}

There's one more trick that's likely useful for low-Mach settings:
\begin{equation}
  \partial_t \vec{u} + \vec{u}\cdot \del\vec{u} + \left(\frac{\gamma}{\gamma-1}\right) \mathrm{Ma}^2 \del\left[h_0 \ln h_1 \right] = \mathrm{Ma}^2 h_0 \exp{(\ln h_1)}\del s -
  \left(\frac{\gamma}{\gamma-1}\right) \mathrm{Ma}^2 \del\left[h_0\left(\exp{(\ln h_1)}-1-\ln h_1\right)\right] + \vec{\scrD}_1\cdot(\nu E),
\end{equation}
or
\begin{equation}
  \partial_t \vec{u} + \vec{u}\cdot \del\vec{u} + \left(\frac{\gamma}{\gamma-1}\right) \mathrm{Ma}^2 \del\left[h_0 \ln h_1 \right] = \mathrm{Ma}^2 h_0 \exp{(\ln h_1)}\del s -
  \left(\frac{\gamma}{\gamma-1}\right) \mathrm{Ma}^2 \del\left[h_0\left(\mathrm{expm1}{(\ln h_1)}-\ln h_1\right)\right] + \vec{\scrD}_1\cdot(\nu E),
\end{equation}
where we've just shifted a likely wave-like linear term to the LHS.  Not clear there is a similar one in the quadratic thermal nonlinearity, but that's also because we assumed $\del s_0 = 0$, and there are no gravity waves in that basic state (just acoustic).  That's actually very interesting.

\subsection{Making a super-adiabatic atmosphere}
If the zero state has $\del s_0 = 0$, then the superadiabaticity must be in the initial conditions.  Here's what that looks like.

The hydrostatically balanced temperature profile is:
\begin{equation}
  T = \frac{g}{c_P} \ldots
\end{equation}

\subsection{Thermal considerations}



The lack of thermal equilibrium implies a constant heating source Q:
\begin{equation}
  Q = \vec{\scrD}_1 \cdot (\chi \ln h_0) + \chi (\del \ln h_0)^2
\end{equation}

The equation set becomes:
\begin{equation}
  \partial_t \vec{u} + \vec{u}\cdot \vec{u} + \del (h_0\left[\exp{(\ln h_1)}-1\right]) = \frac{1}{c_P}h_0(\exp{(\ln h_1)})\del s_1 + \vec{\scrD}_1\cdot(\nu E),
\end{equation}
or
\begin{equation}
  \partial_t \vec{u} + \vec{u}\cdot \vec{u} + \del (h_0 \ln h_1) - \frac{1}{c_P}h_0 \del s_1 = -\del (h_0\left[\exp{(\ln h_1)}-1-\ln h_1 \right]) + \frac{1}{c_P}h_0(\exp{(\ln h_1)} - 1)\del s_1 + \vec{\scrD}_1\cdot(\nu E),
\end{equation}
where we have de-stiffened both the nonlinear pressure gradient term and the nonlinear buoyancy term.
The viscous term is:
\begin{equation}
  \vec{\scrD}_1\cdot(\nu E) = (\del + \del \ln \rho_0 + \del \ln \rho_1) \cdot(\nu E)
\end{equation}
so:
\begin{equation}
  \partial_t \vec{u} + \vec{u}\cdot \vec{u} + \del (h_0 \ln h_1) - \frac{1}{c_P}h_0 \del s_1 - \vec{\scrD}_{1,0} \cdot(\nu E) = -\del (h_0\left[\exp{(\ln h_1)}-1-\ln h_1 \right]) + \frac{1}{c_P}h_0(\exp{(\ln h_1)} - 1)\del s_1 + \del \ln \rho_1\cdot(\nu E),
\end{equation}
We need some better notation.  Let $\Theta = \ln h$ and $\Upsilon = \ln \rho$.
\begin{equation}
  \partial_t \vec{u} + \vec{u}\cdot \vec{u} + \del (h_0 \Theta_1) - \frac{1}{c_P}h_0 \del s_1 - \vec{\scrD}_{1,0} \cdot(\nu E) = -\del (h_0\left[\exp{\Theta_1}-1-\Theta_1 \right]) + \frac{1}{c_P}h_0(\exp{\Theta_1} - 1)\del s_1 + \del \Upsilon_1\cdot(\nu E),
\end{equation}

For the entropy equation, we need to decompose the RHS:
\begin{align}
\vec{\scrD}_1 \cdot (\chi \Theta) + \chi (\del \Theta)^2 &=
\vec{\scrD}_1 \cdot (\chi \Theta_0) + \chi (\del \Theta_0)^2
+ \vec{\scrD}_1 \cdot (\chi \Theta_1) + \chi (\del \Theta_1)^2
+ 2 \chi (\del \Theta_0\cdot \del \Theta_1) \\
& = \vec{\scrD}_{1,0} \cdot (\chi \Theta_0) + \chi (\del \Theta_0)^2 \\
& \phantom{=} + \del \Upsilon_1 \cdot(\chi \Theta_0)
+ \vec{\scrD}_{1,0} \cdot (\chi \Theta_1) \\
& \phantom{=} + \del \Upsilon_1 \cdot(\chi \Theta_1)+ \chi (\del \Theta_1)^2
 + 2 \chi (\del \Theta_0\cdot \del \Theta_1) \\
& = Q + \del \Upsilon_1 \cdot(\chi \Theta_1)+ \chi (\del \Theta_1)^2
+ 2 \chi (\del \Theta_0\cdot \del \Theta_1) \\
& \phantom{=} + \del \Upsilon_1 \cdot(\chi \Theta_0)
+ \vec{\scrD}_{1,0} \cdot (\chi \Theta_1)
\end{align}
the entropy equation is:
\begin{equation}
\frac{1}{c_P}\left(\partial_t s_1 + \vec{u}\cdot \del s_1\right) =
Q +
 \vec{\scrD}_1 \cdot (\chi \ln h) + \chi (\del \ln h)^2 + \exp{(-\ln h)}\frac{\nu}{2}\mathrm{Tr}(E^2).
\end{equation}

Our variables are now:
\begin{equation}
\ln h = \ln h_0 + \ln h_1, \quad ln \rho = \ln \rho_0 + \ln \rho_1, \quad s = s_0 + s_1.
\end{equation}

Thermal equilibrium means:




\end{document}
